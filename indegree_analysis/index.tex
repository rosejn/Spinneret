\documentclass[a4paper,12pt]{article}

\begin{document}

\title{Controlling In-degree in Out-degree Constant P2P Topologies}
\date{\today}
\author{Cyrus P. Hall\\
  hallc@lu.unisi.ch\\
  \\
  Department of Informatics\\
  University of Lugano\\
  %Lugano, Switzerland\\
  %\\
  %Advisor: Antonio Carzaniga
}
\maketitle

\section*{Abstract}
In-degree control in P2P networks has longed been known to be a problem, but
current solutions are less than satisfactory.  We contribute two useful
techniques to help design future structured and unstructured P2P topologies: an
analysis using ergodic markov-models which shows the effects of variable
in-degree, and second, a set of random walk driven transformations to control
in-degree in both structured and unstructured topologies.  Both contributions
are analyzed via simulation.

\section{Introduction}
% include relate work here, including why we feel the multi-node solution is
% not appropriate.

\section{Ergodic Analysis of P2P Topologies}

Ergodic theory has been widely used in the long-term analysis of complex
systems in physics and economics, but has seemingly slipped under the radar of
the P2P research community.  We first offer a quick primer of ergodic
markov-models, and then discuss their application to P2P topologies.  Finally,
we give show an example using a P2P topology we have been developing. 

\subsection{Ergodic Markov-Models}


\subsection{Application of EMMs}
\label{sec:ergodic-application}

\subsubsection{A real world application}

Our own exploration of ergodic theory, and ergodic markov-models in particular,
came about when we noticed our own P2P topology, Spinneret, showed serious
in-degree imbalance.  Spinneret is designed to support both structured and
unstructured search techniques on a single topology, so we were alarmed to see
over an order of magnitude difference between nodes in-degree.  The initial
concern was that poorly balanced in-degree would lead to higher bandwidth usage
for some nodes, creating natural bottlenecks for both greedy and unstructured
search.  However, as we will now show, the analizing the topology as an ergodic
markov-model revealed much a much deeper problem.

\subsection{Common Topology Random-walk Reachability}

It has been claimed that many unstructured techniques should work with minor
modification on structured topologies, such as distributed-hash tables.  After
we discovered the rather radical effects of in-degree on reachability,
particularly in out-degree constant topologies, we came to suspect that these
claims may need to be modified.  As such, we have conducted a brief
investigation of reachability on several commonly studied topologies which,
together, represent a cross-section of P2P topologies: Pastry, Gnutella, and
Newscast/Cyclon.

For each topology, we performed a similar transformation as described in
Section \ref{sec:ergodic-application}.  As neither Pastry or Newscast are
deployed widely on the Internet, we used available simulators and extracted
topologies at various points in the run.  While Gnutella is widely deployed,
there are no complete topology dumps available, so instead we used topology
generators based on the measured properties of live networks.  The sources of
our simulations are available in \cite{1,2,3}.  % Need cites

\section{Controlling In-degree}

Ergodic markov-models, while useful for analyzing topologies, also suggest
possible solutions to control the standard deviation of the distribution of
in-degree.  We now explore one of these.

A direct result of ergodic mixing analysis is the needed length of a walk, $l$,
before it can be considered random to a given degree.  This leads directly to a
strategy for discovery node in-degree without the need for estimation based on
incoming connections and request.  If a constant rate of walks is launched into
the topology, with a time-to-live of at least $l$, the end point of those walks
can be considered a random sample of the visitation rate of nodes in the
network.  According to the results of the ergodic analysis in Section
\ref{sec:ergodic-analysis}, nodes with higher in-degrees have a higher
probability to be at the end of such a random walk, and therefore should have a
higher rate of visitation.  If the number of such random walks in the network
is high enough, the rate of visitation recorded at nodes in the network should
be closely correlated with the actual in-degree of the node.

In fact, we find this to be the case for Spinneret, as Figure
\ref{fig:indegree-visitation-rate} shows.  Pearson's correlation coefficient
gives the correlation between in-degree and visitation as anywhere between, 



\section*{Acknowledgements}
The authors would like to thank Shane Legg from the University of Lugano for
the initial suggestion to explore using ergodic markov-models.  We'd also like
to thank Tiago de Paula Peixoto from the University of São Paulo for his
wonderful graph-tool program, which greatly streamlined our data analysis.

%\bibliographystyle{plain}
%\bibliography{works-cited}

\end{document}
