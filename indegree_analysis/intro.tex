\section{Introduction}

As P2P networks continue to develop and larger deployments are made, it will
become crucial that cheap and effective sampling techniques are available to
measure network dynamics.  While simulation and testbed deployments are
important steps toward creating resilient and well-behaved systems, often
wide-scale deployments expose issues of scaling and dependability not
experienced in the lab.
%Note the recent skype example as a major example of this
It is therefore crucial to develop tools that allow P2P networks and their
maintainers to understand the state of a widely deployed system, allowing for
the introduction of changes in parameterizations or algorithms when necessary.

Similarly, it is often desirable to run unstructured queries across P2P
topologies that were either created for structured search (e.g., DHTs) or that
maintain suboptimal search topologies.  Like the monitoring problems above,
such a search would optimally behave like a \emph{uniform random sample} of
network, with the likelihood of visiting any particular node $1/n$, where $n$
is the number of peers.  Our goal in this paper is to demonstrate such an
algorithm and demonstrate its effectiveness across a range of topologies.

The first attempts at sampling P2P topologies assumed that a random walk was,
as the name suggest, random.  While in a strict sense this is true, the
unstated assumption is that it would be \emph{uniformly} random.  However, P2P
networks (with a few exceptions, such as Cyclon) do not create and maintain
topologies which give uniform visitation probabilities across nodes in the
network.  As such, these methods are biased toward particular nodes, often
those with high in-degree.

Topological bias has largely been solved for undirected networks, but no
comparable solution has been found for unidirectional networks. Unlike
undirected topologies, the probability of reaching a node in a directed
topology is not determined directly by the in-degree.  Rather, a complicated
mixture of in- and out-degree, link-distribution, vertex mixing, clustering,
and more, can all play a role in shaping a nodes final probability of being
reached. In this paper we propose a solution for directed topologies by
augmenting techniques previously used for undirected topologies.

We first discuss an off-line static analysis method involving Markov-chains in
Section \ref{sec:methods}.  Using the presented method we explore various P2P
topologies' reachability behavior.  Further, we also look at mixing-degree, a
parameter that determines the necessary length of a random walk in order that
its destination is chosen uniformly at random. 

Using the knowledge gained in the analysis, we design our random sample
algorithm for directed topologies in Section \ref{sec:sampling}.  The algorithm
has two steps: first, unbiased random walks are used to sample the reachability
of nodes in the network.  These samples are in terms of \emph{frequency of
visitation}, a value directly proportional to the probability of reaching a
node at the end of a random walk.  Using a Metropolized Random Walk with
Backtracking (MRWB) \cite{stutzberg:imc06} and the frequency of visitation
information, we demonstrate how to create biased walks that uniformly sample
the topology.  This technique works universally for any directed topology,
although with varying cost.

Finally, results of the method are given Section \ref{sec:results}.  Although
our technique is rather n\"{a}ive, we demonstrate that it works well across a
wide range of P2P topologies, using the analysis methods developed in Section
\ref{sec:methods}.
